\documentclass[a4paper]{article}
    \title{\vspace{-0.7em}COMS31700: Design Verification - Calculator Design; Coverage Analysis\vspace{-0.7em}}
    \author{Joshua Van Leeuwen - jv15626 - 23305}
\date{}

\usepackage{titlesec}
\usepackage[margin=0.5in]{geometry}
\geometry{top=10mm, bottom=20mm}


\usepackage{titling}
\usepackage{array}
\usepackage{listings}

\setlength{\droptitle}{-5em}

\begin{document}
\vspace{-10em}
\maketitle

\vspace{-5.1em}
\subsection*{Code coverage}
In order to determine whether all branches of the code have been exhausted during testing, code coverage is used.



\subsection*{Collecting Functional Coverage}
In order to ensure that the DUV has been tested thoroughly, functional coverage
is used in conjunction with code coverage to determine all value based events
have been met - not only all code has been executed.

Collecting functional coverage is split into three main blocks that correspond
to the 3 structure types used to test the DUV - serial, parallel of 2 and
parallel of 4 operations. Cross products of the structure fields are used to
test the functional coverage of the tests, ensuring all logical combinations of
inputs have been exhausted. These are the input ports, commands and responses.
Each are cross product with each other as pairs and together. 4 operations in
parallel does not test for port numbers since all 4 ports are used during each
test.

Invalid commands are ignored from all coverage except for crossed with ports
during serial since they do not respond with other responses so only need to
check that they are run on every port at least once. No operation is ignored on
all coverage. Response

When responses are collected after driving tests, coverage data is also
collected.





\end{document}
